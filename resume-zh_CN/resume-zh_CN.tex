% !TEX TS-program = xelatex
% !TEX encoding = UTF-8 Unicode
% !Mode:: "TeX:UTF-8"

\documentclass{resume}
\usepackage{zh_CN-Adobefonts_external} % Simplified Chinese Support using external fonts (./fonts/zh_CN-Adobe/)
% \usepackage{NotoSansSC_external}
% \usepackage{NotoSerifCJKsc_external}
% \usepackage{zh_CN-Adobefonts_internal} % Simplified Chinese Support using system fonts
\usepackage{linespacing_fix} % disable extra space before next section
\usepackage{cite}

\begin{document}
\pagenumbering{gobble} % suppress displaying page number

\name{储~~~~子~~~~悦}

\basicInfo{
  \email{zero-cooper@foxmail.com} \textperiodcentered\ 
  \phone{(+86) 177-510-34681} \textperiodcentered\ 
  \linkedin[Zero Cooper]{www.linkedin.com/in/zerocooper}}

\section{\faGraduationCap\  教育背景}
\datedsubsection{\textbf{南京航空航天大学}, 江苏,南京}{2014 -- 2018}
\textit{本科毕业生}\ 自动化专业

\datedsubsection{\textbf{University of Edinburgh}, Edinburgh,UK}{2018 -- 2019}
\textit{在读研究生}\ 人工智能专业

\section{\faUsers\ 实习/项目经历}
\datedsubsection{\textbf{中国石化仪征化纤股份有限责任公司},仪征}{2017年6月 -- 2017年8月}
\role{技术助理}{实习}
\begin{itemize}
  \item 学习了分布式离散系统(DSC)的原理及其组态
  \item 部分参与了DSC组态的编程,包括中央控制室I/O与工作室标准状态的编程
  \item 在中央控制室中对PET生产过程进行实时监控
\end{itemize}

\datedsubsection{\textbf{基于树莓派的智能互联网监控系统设计}}{2017年11月 -- 2018年1月}
\role{Raspberry Pi, Python, opencv}{团队项目}
\begin{onehalfspacing}
智能远程监控系统中人脸识别部分的设计
\begin{itemize}
  \item 设计面向视屏流的人脸识别的基本算法,并通过Adaboost原理对原算法进行优化提升
  \item 正脸识别率达90\%,响应时间降低了一半,弱化了系统对环境的依赖度

\end{itemize}
\end{onehalfspacing}

\datedsubsection{\textbf{智能平衡车平台的开发}}{2017 年2月 -- 2018年6月}
\role{校级资金支持项目,PID控制,Matlab}{团队项目}
\begin{onehalfspacing}
设计智能平衡车平台的控制算法
\begin{itemize}
  \item 理论上建立了智能平衡车的运动的数理方程,并利用二阶倒立摆的原理建立其控制关系
  \item 建立了基于不同路面状态下的模糊逻辑表,并利用模糊PID控制对模糊逻辑表进行调参
  \item 撰写了论文《小波分析在间断点检测与阈值消噪方面的应用》(待发表)
\end{itemize}
\end{onehalfspacing}

% Reference Test
%\datedsubsection{\textbf{Paper Title\cite{zaharia2012resilient}}}{May. 2015}
%An xxx optimized for xxx\cite{verma2015large}
%\begin{itemize}
%  \item main contribution
%\end{itemize}

\section{\faCogs\ IT 技能}
% increase linespacing [parsep=0.45ex]
\begin{itemize}[parsep=0.4ex]
  \item 编程语言: C++ == C > Matlab > Python > \LaTeX > HTML > 汇编
  \item 办公: 熟练的Microsoft Office使用者,包括Word, Excel, Powerpoint, Visio

\end{itemize}

\section{\faHeartO\ 获奖情况}
\datedline{南京航空航天大学奖学金}{2014 \& 2015 \& 2016 \&2017 年}
\datedline{南京航空航天大学青年物理学家竞标赛二等奖}{2017年}
\datedline{上海模拟联合国大会杰出代表奖}{2016年}
\datedline{南京航空航天大学自动化学院辩论赛最佳辩手}{2016年}
\datedline{全国大学生英语能力竞赛二等奖}{2015年}
\datedline{第12届中国模拟联合国大会荣誉提名奖}{2015年}

\section{\faInfo\ 社会经历及其他}
% increase linespacing [parsep=0.5ex]
\begin{itemize}[parsep=0.4ex]
  \item \datedline{UMUNC2017西南分会GAUS体系学术指导}{2017 年}
  \item \datedline{南京航空航天大学模拟联合国协会副会长(一等学生干部)}{2015 年}
  \item 语言:英语-熟练(CET-4:581、CET-6:542、IELTS:6.5)

\end{itemize}

%% Reference
%\newpage
%\bibliographystyle{IEEETran}
%\bibliography{mycite}
\end{document}
